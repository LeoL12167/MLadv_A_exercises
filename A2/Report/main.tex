\documentclass{article}
\usepackage[utf8]{inputenc}
\usepackage{geometry}
\usepackage[T1]{fontenc}
\usepackage{amsfonts}
\usepackage{amsmath}
\usepackage{graphicx}
\usepackage{float}
\usepackage{hyperref}
\usepackage[sorting=none]{biblatex}
\usepackage{fancyhdr}
\usepackage{multicol}
\usepackage{enumitem}
\addbibresource{ref.bib}
\setlength{\columnsep}{40pt}
\setlength{\voffset}{0.7cm}
\setlength{\headsep}{40pt}
\geometry{legalpaper, portrait, margin=2cm}


% Title page
\title{HW A2\\\Large{Machine Learning, Advanced Course/DD2434/mladv24}}
\author{Aurhor \\ KTH Royal Institute of Technology\\ School of Engineering Sciences in Chemistry, Biotechnology and Health}
\date{\today}

% Header and footer
\pagestyle{fancy}
\fancyhead{}
\fancyhead[L]{\textbf{Machine Learning, Advanced Course}\\\textbf{DD2434}}
\fancyhead[R]{\textbf{Andrea Stanziale, Leonardo Lüder}\\ stanz@kth.se, luder@kth.se}
\fancyfoot{}
\begin{document}

\maketitle

% Begin page numbers
\fancyfoot[C]{\thepage}
\pagenumbering{arabic}
\begin{multicols}{2}

    \section*{2.1}
    The answers to the questions are as follows:
    \begin{enumerate}[noitemsep, topsep=0pt]
        \item Bernoulli
        \item Gamma
        \item Log-normal
        \item Beta
    \end{enumerate}
    \section*{2.2}
    \subsection*{2.2.1} 
    The definition given of the local hidden variables is that they must be dependent
    only on other local variables of the local context and the global variables, that is:
    \begin{align}
        p(z_{d_i,n} = k |\theta_d, \beta, w_{d,n}) = p(z_{d_i,n} = k | \theta_{d_i}, \beta, w_{d_i,n})
    \end{align}
    \subsection*{2.2.2}

    \subsection*{2.2.3}
    


\section*{2.3 BBVI}
In BBVI without Rao-Blackwellization and control variates, the gradient is estimated using Monte Carlo sampling, the score function of \(q\), and the joint of \(p\).


\subsection*{Question 2.3.9}
Let \(X = (X_1, \dots, X_N)\) be i.i.d. with \(X_n | \theta, \sigma^2 \sim \mathcal{N}(\theta, \sigma^2)\), \(\theta \sim \text{LogNormal}(\nu, \epsilon^2)\), and \(\sigma^2\) fixed. Write an expression for the Naive BBVI gradient estimate w.r.t. \(\alpha\) using one sample \(z \sim q(\theta)\), \(q(\theta) = \text{Gamma}(\alpha, \beta)\). (2 points)


\subsection*{Question 2.3.10}
Describe in one sentence what Rao-Blackwellization is used for in the BBVI paper. (1 point)

\subsection{Question 2.3.11}
Given the model in figure 2, and the mean-field approximation:  
\[
q(y, w, z, v) = q_\lambda^1(w)q_\lambda^2(z)q_\lambda^3(v)\prod_n q_\lambda^4(y_n),
\]
describe qualitatively how the Rao-Blackwellized partial gradient of the ELBO w.r.t. \(\lambda_n^4\), \(\nabla_{\lambda_n^4} \mathcal{L}\), is obtained. Write out the final expression for the Rao-Blackwellized \(\nabla_{\lambda_n^4} \mathcal{L}\). (2 points)







\end{multicols}

\clearpage
\addcontentsline{toc}{section}{References}
% \printbibliography{}

\end{document}
