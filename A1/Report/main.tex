\documentclass{article}
\usepackage[utf8]{inputenc}
\usepackage{geometry}
\usepackage[T1]{fontenc}
\usepackage{amsfonts}
\usepackage{graphicx}
\usepackage{float}
\usepackage{hyperref}
\usepackage[sorting=none]{biblatex}
\usepackage{fancyhdr}
\usepackage{multicol}
\addbibresource{ref.bib}
\setlength{\columnsep}{40pt}
\setlength{\voffset}{0.7cm}
\setlength{\headsep}{40pt}
\geometry{legalpaper, portrait, margin=2cm}


% Title page
\title{Essay title\\\Large{Machine Learning, Advanced Course/DD2434/mladv24}}
\author{Aurhor \\ KTH Royal Institute of Technology\\ School of Engineering Sciences in Chemistry, Biotechnology and Health}
\date{\today}

% Header and footer
\pagestyle{fancy}
\fancyhead{}
\fancyhead[L]{\textbf{Machine Learning, Advanced Course}\\\textbf{DD2434}}
\fancyhead[R]{\textbf{Andrea Stanziale, Leonardo Lüder}\\ stanziale@kth.se, luder@kth.se}
\fancyfoot{}
\begin{document}

\maketitle
\thispagestyle{fancy}
\clearpage
\tableofcontents
\thispagestyle{fancy}

\clearpage
% Begin page numbers
\fancyfoot[C]{\thepage}
\pagenumbering{arabic}
\begin{multicols}{2}

\section*{Introduction}
\addcontentsline{toc}{section}{Introduction} % For the contents page

Name1 Surname1, Name2 Surname2, Name3 Surname3 
KTH Royal Institute of Technology, School of ... 
ackmakcaksmcksdmc
 
Contributions of the authors: 
Name1 Surname1 made 10 cups of coffee. 
Name2 Surname2 typed the text 
Name3 Surname3 accumulated the reference list 



The introduction summarizes background knowledge for your project. It should contain quite 
a bit more than one would find in a scientific paper since it’s important that your reviewers 
and the examiner can understand what your project is about. 



Subdividing the introduction into different topics can add structure to the report. 
The same is true for the following two sections. 



Length and layout of the report 
In total the report shouldn't contain more than about 10 pages of text. This does not include 
figures, tables and equations. The reference list is not counted as text either. 
The text should be written single spaced in some legible size 12 font, for example Times. 



A citation, uses the ref.bib file.\cite{Beauchamp2015}
Recommend \href{https://www.doi2bib.org/}{https://www.doi2bib.org/} for getting .bib style references to insert into the ref.bib file.

\section*{Materials and Methods}
\addcontentsline{toc}{section}{Materials and Methods}

This section contains only information about what was done, how it was done and with what 
instrumentation, algorithms, computer programs, chemicals and so on. 
No results should be discussed here and nothing should be concluded. 
It’s important that every author contributes to the project in a well defined way. 
The contribution of each author has to be stated in detail on the first page of the report (see 
above) and each author is expected to present at least their contribution and also answer 
questions from the reviewers. 
It’s best to use Equation Editor to write equations if you use Word. If you use another text 
editor, such as Latex, you’ll know better than me what to do. 
Equations that you refer to in the text should be numbered. 


\section*{Results (and discussion)}
\addcontentsline{toc}{section}{Results (and discussion)}

This is where the results are presented and can also be discussed, but discussion can also be 
kept for the next section. Do what seems more natural for your text. 
Figures and tables should be numbered and have captions and have to be referred to and 
explained a bit more in the text. 
A figure caption is located below the figure a table caption above the table. 
Use something like Endnote to organize your references and format them as (name, year) in 
the text so that they are in alphabetical order in the reference list. 


\paragraph{Paragraphs} are often helpful.

\section*{Discussions (and conclusions)}
\addcontentsline{toc}{section}{Discussions (and conclusions)}

Here you discuss your results and their limitations and present conclusions if possible.

\end{multicols}
\clearpage
\addcontentsline{toc}{section}{References}
\printbibliography

\end{document}
